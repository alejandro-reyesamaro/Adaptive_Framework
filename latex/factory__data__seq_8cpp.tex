\hypertarget{factory__data__seq_8cpp}{\section{src/factory\-\_\-data\-\_\-seq.cpp \-File \-Reference}
\label{factory__data__seq_8cpp}\index{src/factory\-\_\-data\-\_\-seq.\-cpp@{src/factory\-\_\-data\-\_\-seq.\-cpp}}
}


\-Implementation of the \-Factory \-Method pattern to build (sequential) \hyperlink{classAdData}{\-Ad\-Data}.  


{\ttfamily \#include \char`\"{}factory\-\_\-data\-\_\-seq.\-hpp\char`\"{}}\*


\subsection{\-Detailed \-Description}
\-Implementation of the \-Factory \-Method pattern to build (sequential) \hyperlink{classAdData}{\-Ad\-Data}. \-Adaptive \-Framework

\begin{DoxyAuthor}{\-Author}
\-Florian \-Richoux 
\end{DoxyAuthor}
\begin{DoxyDate}{\-Date}
2013-\/02-\/04 
\end{DoxyDate}
